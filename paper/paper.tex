\documentclass{llncs}

\usepackage{amsmath}
\PassOptionsToPackage{hyphens}{url}
\usepackage{hyperref}

\usepackage[backend=biber,bibencoding=utf8,urldate=iso8601,date=iso8601]{biblatex}

\addbibresource{mybib.bib}

\urlstyle{sf}

\usepackage{todonotes}

\definecolor{aliceblue}{rgb}{0.94, 0.97, 1.0}

\newcommand{\as}[1]{\todo[color=green!20]{#1}}
\newcommand{\asi}[1]{\todo[color=green!20,inline]{#1}}
\newcommand{\mm}[1]{\todo[color=aliceblue]{#1}}
\newcommand{\mmi}[1]{\todo[color=aliceblue,inline]{#1}}

\newcommand{\jsinline}[1]{\texttt{#1}}

\usepackage{xspace}

\usepackage{minted}

\begin{document}

\title{Certified Desugaring of Javascript Programs using Coq}

\author{Marek Materzok\inst{1} \and Alan Schmitt\inst{2}}
\institute{University of Wroc\l{}aw, Poland\\\email{marek.materzok@ii.uni.wroc.pl}
\and Inria, France\\\email{alan.schmitt@inria.fr}}

\maketitle

\newcommand{\lambdajs}{$\lambda_\textrm{JS}$\xspace}

\begin{abstract}
JavaScript is a programming language originally developed for client-side
scripting in Web browsers; its use evolved from simple scripts to
complex Web applications. It has also found use in mobile applications,
server-side network programming, and databases.

A number of semantics were developed for the JavaScript language.
We are specifically interested in two of them: JSCert and \lambdajs.
In order to increase our confidence that the two semantics correctly
model JavaScript, we relate them formally using Coq. The size and complexity
of the two semantics makes this a complex problem with many obstacles
to be overcome.
\end{abstract}

\section{Notes}

\begin{itemize}
\item give figures on test coverage
\item say we created tests for bugs
\item say thanks to Valentin Lorentz, to the people at Brown
\item related work: KJS
\end{itemize}

\section{Introduction}

The JavaScript language is standardized by ECMA under the name ECMAScript.
Several versions of the specification now exist, the latest ones being
ECMAScript 5.1 (ES5, released in 2011), ECMAScript 2015, and ECMAScript
2016.\footnote{The ECMAScript standard is now published yearly.} The role of the
specification is to make portability between different JavaScript
implementations possible by specifying the intended behavior of the various
language constructs and built-in functions of JavaScript. The specification also
makes it possible to develop formal semantics for the language, which can then
be used to, e.g., develop static analysis tools or guarantee that the invariants
required by the specification can be satisfied.

One such semantics is \lambdajs. It aims to capture
the semantics of JavaScript by defining a simple core language
(based on the untyped call-by-value lambda calculus), and then
transforming (or ``desugaring'') the full JavaScript language into this
simple core. The \lambdajs semantics was originally developed to model the 
ECMAScript 3 specification~\cite{Guha-al:ECOOP10}, and it was later
revised~\cite{Politz-al:DLS12} to handle ECMAScript 5.1
features.\footnote{The authors named the revised language S5; however,
we continue to name it \lambdajs for simplicity.}
The interpreter for the core language and the JavaScript-to-core
transformation were developed using 
OCaml~\cite{github:brownplt/LambdaS5};
the interpreter was tested using the official test262 test suite.
The \lambdajs semantics is useful for tool developers, as it
allows the tools to analyze JavaScript programs by working
on the simple core language of \lambdajs, having all the complexities
of JavaScript handled in the desugared code. It has already been
successfully used to verify ADsafe~\cite{Politz-al:SEC11}.

More recently, JSCert~\cite{Bodin-al:POPL14}, formalized the ECMAScript 5.1
specification in the form of a pretty-big-step
semantics~\cite{Chargueraud:ESOP13} expressed in the form of a Coq inductive
predicate. It aims to be as close to the (informal English) specification as
possible by maintaining close correspondence between the evaluation rules and
steps in the ES5 pseudocode. The project also includes JSRef, a reference
interpreter for JavaScript written in Coq and proven correct with respect to
JSCert. The interpreter is also tested using the official ECMA test suite,
test262. Thanks to the correctness proof, testing of JSRef increases the
confidence that JSCert faithfully models ES5.

The goal of this work is to formally relate the two semantics. Our motivation
is twofold. First, as the two semantics use a different approach, relating them
increases the confidence we can place in both of them, and has the side effect
of uncovering inconsistencies or other problems. Second, a formal and certified
version of \lambdajs, obtained by proving its correctness with respects to
JSCert, enables development of certified analysis tools using \lambdajs as the
core.

The main challenge lies in the complexities of the JavaScript language. The ES5
specification, upon which JSCert and \lambdajs are based, is 258 pages long and
has 16 chapters. The JSCert pretty-big-step semantics, which models the
specification, has over 900 rules; the entire Coq source code for the semantics
has over 10000 lines. The \lambdajs environment file, which contains the
ECMAScript algorithms expressed in core \lambdajs code, is almost 5000 lines
long.

As is done in \lambdajs, we focus on the \emph{strict mode} restriction of
JavaScript. Our contributions are as follows.
\begin{enumerate}
\item Two formalizations in Coq of \lambdajs: an inductive one, and an
  executable one. The formalizations coincide: they are shown to be correct and
  complete in relation to each other. This contribution is described in
  Section~\ref{sec:form-lambdajs}.
\item A formalization in Coq of desugaring from JavaScript to \lambdajs, as well
  as the initial \lambdajs environment in Coq. This contribution is described in
  Section~\ref{sec:desugaring}.
\item A correctness proof of the desugaring, stating that if a JavaScript
  program translated to \lambdajs successfully terminates, then the initial
  program has a semantics according to JSCert and the results match. This is
  described in Section~\ref{sec:corr-proof-lambdajs}.
\end{enumerate}

Note that we do not prove the converse, i.e., that if a JavaScript program has a
semantic in JSCert, then its \lambdajs translation successfully terminates and
the results match. For such a result, we would need to apply \textsf{inversion}
on the full JSCert semantics, which does not currently work. This issue is
described in Section~\ref{sec:coq-techniques} alongside the techniques we
implemented in Coq for this development.

The development is publicly available and can be downloaded from
GitHub~\cite{github:lambdacert}.

\section{Formalization of \lambdajs}
\label{sec:form-lambdajs}

Because of the various issues we found in \lambdajs as defined
in~\cite{Politz-al:DLS12} and which we detail in Section~\ref{sec:issues},
we decided to revise the language. The full list of changes is
detailed in~TODO\mm{Can we just link the wiki page? Or should we use an
  appendix?}\as{Let's do both: put them in the appendix, and link to the wiki.}, we present some of them below.
\begin{itemize}
\item The attribute access and modification operators, which tried to directly
    model the operators from JavaScript, were removed. We found that the complexities
    of attribute access is much better handled in the desugaring.
\item We removed the imperative features of the language which do not involve the
    object heap. This change simplified the definition of the semantics, which made
    the proofs easier.
\item We added a new primitive value named \texttt{empty}, which models the empty
    completion value of JavaScript, and a second kind of the semicolon operator,
    which returns the last non-empty value of its subexpressions. This allowed to
    correctly implement statement completion values in a natural way.
\item An integer datatype was added. Even though ES5 itself has no integers,
    they are used in the definition of the binary operators (bitwise logical operators,
    bit shifting).
\end{itemize}

For the purpose of relating the semantics of \lambdajs and ES5, we developed a
formalization of the \lambdajs core language. This was necessary because no such
formalization existed previously; a formalization of the original, ES3-based
\lambdajs was available\cite{github:brownplt/LambdaJS}, but the two
versions of \lambdajs differ too much for this to be useful.

We decided to formalize the \lambdajs core using a
pretty-big-step semantics, as in JSCert, to make the job of
relating the two semantics easier. We have also developed
an interpreter for \lambdajs, which we proved sound and
complete with respect to the pretty-big-step 
semantics.\footnote{The initial version of the interpreter was
written by Martin Lorentz.}
The proof script was developed using domain-specific tactics,
which made it simple and robust. They are detailed in
Section~\ref{subsec:ljstactics}.

\subsection{Coq Formalization}
\label{subsec:ljscoq}

\asi{starting from the main lemmas, describe the Coq formalization, describing
  necessary details as they occur.}

\begin{minted}{coq}
Inductive resultof (T : Type) : Type :=
| result_some : T -> resultof T
| result_fail : string -> resultof T        (* program error *)
| result_impossible : string -> resultof T  (* inconsistency *)
| result_bottom : resultof T                (* out of fuel *)
| result_dump : ctx -> store -> resultof T. (* dump state *)

Definition eval_S eval c st (e : expr) : result :=
  let return_value := result_value st in
  match e with
  | expr_empty => return_value value_empty
  | expr_undefined => return_value value_undefined
...

Definition eval_0 : eval_fun := fun c st _ => result_bottom.

Fixpoint eval_k max_step : eval_fun :=
  match max_step with
  | 0 => eval_0
  | S max_step' => eval_S (eval_k max_step')
  end
.

Lemma eval_k_complete : forall c st e o,
  red_expr c st e o -> 
  exists k, eval_k k c st e = result_some o.

Record eval_fun_correct eval := forall c st e o,
  eval c st e = result_some o -> red_expr c st e o.

Lemma eval_k_correct : forall k, runs_type_correct (eval k). 
\end{minted}

\asi{Add the theorem.}

\subsection{Tactics}
\label{subsec:ljstactics}

\asi{Describe the tactics used in the \lambdajs proofs.}

\section{Desugaring}
\label{sec:desugaring}

We have also implemented in Coq the other two crucial components
of the \lambdajs semantics: the desugaring function and the 
environment. The desugaring was implemented as Coq functions,
taking JavaScript statements and expressions (as defined in
JSCert) to \lambdajs expressions. The environment, which
implements the complexities of the JavaScript semantics
in \lambdajs code, was translated from \lambdajs syntax
to Coq definitions with the help of an OCaml parser and
our interpreter. The resulting Coq file is over 15000 lines long
and takes almost a megabyte of disk space!

\asi{we should put a code snippet of the environment here. Also, do we want to
  talk more about the OCaml tool used to generate it?}

\section{Correctness proof for \lambdajs desugaring}
\label{sec:corr-proof-lambdajs}

\asi{This was in the introduction, we need to integrate it again.

We were able to prove correctness of a large part of the
JavaScript semantics, including function calls, lexical environment
handling, and most operations involving objects. The work is still
in progress to prove the remaining parts, which include integer
operations, the
defineOwnProperty method, and elements of the standard library.
}


In order to establish correctness, it was first needed to create
a set of predicates specifying the relationship between 
JavaScript and \lambdajs language features. As \lambdajs was
designed to implement JavaScript, the relationship is fairly
simple and natural. One place where the two languages differ
is the object heap, which is used in \lambdajs to implement
several JavaScript language features, including the
JavaScript lexical environment. To manage this, we use a bisimulation
relation, which relates JavaScript objects and environment records
to \lambdajs objects.

For the proof, we also need to keep track of certain invariants, which state
that, e.g., the JavaScript objects and their \lambdajs representations are
correctly formed. The invariants are split into two relations, where one
involves the evaluation contexts, and the other the heaps. Both of these involve
the bisimulation relation, but the context invariant uses it only positively. As
the bisimulation relation only grows during the evaluation (as new objects are
created but objects are never collected), this makes it possible to keep the
proofs simple.

The correctness proof is structured using well-founded 
induction on the depth of the \lambdajs pretty-big-step
semantics derivation. This grants a lot of freedom in structuring
the proof, which is important because of certain structural
differences between JSCert and \lambdajs, which involve,
e.g., handling of ES5 references. We use three kinds of induction
hypotheses: for JavaScript statements, expressions and
function calls. The separate induction hypothesis for function
calls was required because of the complexity of calling functions
in JavaScript, and also because function calls can happen in
many possible places in JavaScript, including every
object to primitive conversion.

We prove many lemmas about the different language constructs
of JavaScript, including the internal ``specification functions''
(e.g., type conversions, operations on objects and environment
records, etc.). Typically, the proof for a lemma consists of
a series of forward reasoning steps and case analyses, and ends with
constructing a JSCert semantics derivation and proving that
the invariants were preserved. The forward reasoning is largely
automated using domain-specific tactics; the JSCert derivation
is constructed using the \texttt{eauto} tactic with a specialized
hint database (\texttt{nocore} is used to avoid using the standard
hints coming from Coq and Chargu\'eraud's TLC
library~\cite{TLC}, which is
used both by JSCert and our development).

\section{Coq techniques}
\label{sec:coq-techniques}

Proving the other direction requires doing an inversion of the full JSCert
semantics. Unfortunately, the techniques we have used do not scale to semantics
of that size.

\subsection{Precomputed inversion}

Developing the correctness proof for the JavaScript desugaring required
inverting the \lambdajs pretty-big-step semantics judgment; this
turned out to be very slow, and also generated large proof terms, which 
increased the memory footprint. The problem is well-known~\cite{Monin:Coq2}.
To solve it, we precomputed the required inversion principles
using \texttt{Derive Inversion}, e.g.:
\begin{minted}{coq}
Derive Inversion inv_red_exprh_if with (forall k c st e e1 e2 oo,
    red_exprh k c st (expr_if e e1 e2) oo) Sort Prop.
\end{minted}
Using these, we created a specialized inversion tactic
which selects and applies the required induction principle.
The \texttt{Derive Inversion} calls and the tactic definitions
are generated automatically from the Coq definition of the semantics.

Unfortunately, this technique was still not efficient enough to be
applied to JSCert: the definition of the JavaScript semantics in
JSCert has around 10 times more semantic rules than the definition
in \lambdajs.

\subsection{Proof automation}

\section{Issues found}
\label{sec:issues}

\subsection{Issues in the \lambdajs definition}

In the course of our work, we found numerous cases where
\lambdajs did not faithfully follow the ES5 
specification.\footnote{\url{https://github.com/tilk/LambdaCert/wiki/Issues-with-LambdaS5}}
Some of these issues were serious enough that
far-reaching changes in the \lambdajs environment were needed
to solve them. Interestingly, no serious issues in JSCert were
found, which strengthens our confidence that it accurately models
the ES5 specification.

\subsection{An issue in the ES5 specification}

While proving the correctness of while loops, we found an interesting
inconsistency between the ES5 specification and the behavior
of the two most popular JavaScript implementations, Google's V8 and
Mozilla's SpiderMonkey. The issue involves a subtle interaction
between the \jsinline{break} statements, looping constructs
and statement completion values (which can be observed only by \jsinline{eval}).
ES5 specifies that if the loop body breaks to a~label outside the loop,
the completion values returned by the loop body in the previous iterations
are forgotten. For instance, ES5 specifies the following code to return~1 when run by \jsinline{eval}:
\begin{minted}{javascript}
var x; 
l: { 
    x = 1; 
    while(true) if (x) x = x - 1; else break l 
}
\end{minted}
This kind of semantics is problematic because it breaks the loop unrolling
transformation. If we unroll the \jsinline{while} loop once, we get the
following code, which ES5 specifies to return~0:
\begin{minted}{javascript}
var x; 
l: { 
    x = 1; 
    if (true) { 
        if (x) x = x - 1; else break l; 
        while(true) if (x) x = x - 1; else break l 
    }
}
\end{minted}
The two pieces of code, when run by the major JavaScript engines, both return~0.
The discovery of the inconsistency
resulted in a correction being made in the official ES6
specification\cite{esdiscuss:loop-unrolling}.

\subsection{Issues in JSCert}

\mmi{TODO}

\section{Conclusion}

The proof scripts we developed are very taxing for the Coq
proof assistant -- the compilation of the whole development
requires several gigabytes of RAM memory, and takes several
hours to compile, even though efforts were made to optimize it.
We are still looking for a solution for the performance issues.

The development is still not complete, but it already
includes most of the important core features of the
JavaScript language. We are looking forward to receiving
feedback from the community.

\printbibliography

\end{document}

%%% Local Variables:
%%% coding: utf-8
%%% TeX-engine: xetex_sh
%%% End:
