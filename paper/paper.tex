%-----------------------------------------------------------------------------
%
%               Template for sigplanconf LaTeX Class
%
% Name:         sigplanconf-template.tex
%
% Purpose:      A template for sigplanconf.cls, which is a LaTeX 2e class
%               file for SIGPLAN conference proceedings.
%
% Guide:        Refer to "Author's Guide to the ACM SIGPLAN Class,"
%               sigplanconf-guide.pdf
%
% Author:       Paul C. Anagnostopoulos
%               Windfall Software
%               978 371-2316
%               paul@windfall.com
%
% Created:      15 February 2005
%
%-----------------------------------------------------------------------------


\documentclass{sigplanconf}

% The following \documentclass options may be useful:

% preprint      Remove this option only once the paper is in final form.
% 10pt          To set in 10-point type instead of 9-point.
% 11pt          To set in 11-point type instead of 9-point.
% authoryear    To obtain author/year citation style instead of numeric.

\usepackage{amsmath}
\PassOptionsToPackage{hyphens}{url}
\usepackage{hyperref}
\usepackage{listings,lstlangcoq}

\urlstyle{sf}

\begin{document}

\special{papersize=8.5in,11in}
\setlength{\pdfpageheight}{\paperheight}
\setlength{\pdfpagewidth}{\paperwidth}

\conferenceinfo{COQPL '16}{January 23, 2015, St. Petersburg, Florida, United States} 
\copyrightyear{2016} 
\copyrightdata{978-1-nnnn-nnnn-n/yy/mm} 
\doi{nnnnnnn.nnnnnnn}

% Uncomment one of the following two, if you are not going for the 
% traditional copyright transfer agreement.

%\exclusivelicense                % ACM gets exclusive license to publish, 
                                  % you retain copyright

\permissiontopublish             % ACM gets nonexclusive license to publish
                                  % (paid open-access papers, 
                                  % short abstracts)

\titlebanner{banner above paper title}        % These are ignored unless
\preprintfooter{short description of paper}   % 'preprint' option specified.

\title{Certified Desugaring of Javascript Programs using Coq}
\subtitle{Experience Report}

\authorinfo{Marek Materzok}
           {INRIA, University of Wroc\l{}aw}
           {marek.materzok@inria.fr}

\maketitle

\newcommand{\lambdajs}{$\lambda_\textrm{JS}$}

\begin{abstract}
JavaScript is a programming language originally developed for client-side
scripting in Web browsers; its use evolved from simple scripts to
complex Web applications. It has also found use in mobile applications,
server-side network programming, and databases.

A number of semantics were developed for the JavaScript language.
We are specifically interested in two of them: JSCert and \lambdajs{}.
In order to increase our confidence that the two semantics correctly
model JavaScript, we try to relate them formally using Coq. The size and complexity
of the two semantics makes this a complex problem with many obstacles
to be overcome.
\end{abstract}

\category{D.3.1}{Programming Languages}{Formal Definitions and Theory}

% general terms are not compulsory anymore, 
% you may leave them out
%\terms
%term1, term2

\keywords
Coq, JavaScript, ECMAScript, mechanized semantics

\section{Notes}

\begin{itemize}
\item give figures on test coverage
\end{itemize}

\section{Introduction}

The JavaScript language is standardized by ECMA under the name
ECMAScript. Several versions of the standard now exist, the latest
two being ECMAScript 5.1 (ES5, released in 2011) and ECMAScript 6
(ES6, released in 2015). Its role is to make portability between different
JavaScript implementations possible by specifying the intended behavior
of the various language constructs and built-in functions of JavaScript.
The specification also makes it possible to develop formal semantics
for the language, which can then be used for, e.g., developing
static analysis tools.

One such semantics is JSCert~\cite{Bodin-al:POPL14}, which has the
goal of formalizing the ECMAScript (version 5.1) specification in the form of
a pretty-big-step semantics~\cite{Chargueraud:ESOP13} expressed in 
the form of a Coq inductive predicate. It aims to be as close to the
(informal English) specification as possible by maintaining close correspondence
between the evaluation rules and steps in the ES5 pseudocode. The project
also includes JSRef, a reference interpreter for JavaScript written in Coq
and proven correct with respect to JSCert. The interpreter is also tested
using the official ECMA test suite, test262. Thanks to the correctness proof,
testing of JSRef increases the confidence we can have that JSCert
correctly models ES5.

Another semantics we are interested in is \lambdajs. It aims to capture
the semantics of JavaScript by defining a simple core language
(based on the untyped call-by-value lambda calculus), and then
transforming (or ``desugaring'') the full JavaScript language into this
simple core. The \lambdajs{} semantics was originally developed to model the 
ECMAScript 3 specification~\cite{Guha-al:ECOOP10}, and it was later
revised~\cite{Politz-al:DLS12} to handle ECMAScript 5.1
features\footnote{The authors named the revised language S5; however,
we continue to name it \lambdajs{} for simplicity.}.
The interpreter for the core language and the JavaScript-to-core
transformation were developed using 
OCaml\footnote{\url{https://github.com/brownplt/LambdaS5}};
the interpreter was tested using the official test262 test suite.
The \lambdajs{} semantics could be useful for tool developers, as it
allows the tools to analyze JavaScript programs by working
on the simple core language of \lambdajs, having all the complexities
of JavaScript handled in the desugared code. It was already
successfully used to verify AdSafe. %TODO citation

We believe is worthwhile to formally relate the two semantics. One reason is,
because the two semantics use a different approach, relating
them will increase the confidence we can place in both of them,
and is likely to find inconsistencies or other problems. Another reason is
that having \lambdajs{} formalized and certified (by proving correctness
with respects to JSCert) enables development of certified analysis tools
using \lambdajs{} as the core.

This is easier said than done. The task is very complex, mostly due
to the complexities of the JavaScript language. The ES5 specification,
upon which JSCert and \lambdajs{} are based, is 258 pages long and 
has 16 chapters. The JSCert pretty-big-step semantics, which models
the specification, has over 900 rules; the entire Coq source code
for the semantics has over 10000 lines. The \lambdajs{} environment
file, which contains the ECMAScript algorithms expressed in core
\lambdajs{} code, is almost 5000 lines long.

To reduce the complexity of the task, we are currently working
only on proving \lambdajs{} correct with respect to JSCert,
i.e., that if the \lambdajs{} program successfully terminates,
then its result matches the result given by JSCert. We are also
restricting our attention to ECMAScript strict mode, proving the
correctness of non-strict features only when it is convenient.

We were able to prove correctness of a large part of the
JavaScript semantics, including function calls, lexical environment
handling, and most operations involving objects. The work is still
in progress to prove the remaining parts, which include integer
operations, the
defineOwnProperty method, and elements of the standard library.

The development is publicly available and can be downloaded from GitHub:

\url{https://github.com/tilk/LambdaCert}

\section{Formalization of \lambdajs{}}

For the purpose of relating the two semantics, we first
developed a formalization of the \lambdajs{} core language.
This was necessary because no such formalization existed previously;
a formalization of the original, ES3-based \lambdajs{}
was available\footnote{\url{https://github.com/brownplt/LambdaJS}},
but the two versions of \lambdajs{} differ too much for this
to be useful.

We decided to formalize the \lambdajs{} core using a
pretty-big-step semantics, as in JSCert, to make the job of
relating the two semantics easier. We have also developed
an interpreter for \lambdajs{}, which we proved sound and
complete with respect to the pretty-big-step semantics.
The proof script was developed using domain-specific tactics,
which made it simple and robust.

At this stage we stumbled upon an interesting performance issue 
involving the Coq proof assistant. The proof of completeness required
inverting the pretty-big-step semantics judgment; this turned out
to be very slow, and also generated large proof terms, which increased
the memory footprint. The problem is well-known~\cite{Monin:Coq2}; 
our solution involved
precomputing the required inversion principles using
\texttt{Derive Inversion}, and using the precomputed inversion
principles to create a specialized inversion tactic.

We have also implemented in Coq the other two crucial components
of the \lambdajs{} semantics: the desugaring function and the 
environment. The desugaring was implemented as Coq functions,
taking JavaScript statements and expressions (as defined in
JSCert) to \lambdajs{} expressions. The environment, which
implements the complexities of the JavaScript semantics
in \lambdajs{} code, was translated from \lambdajs{} syntax
to Coq definitions with the help of an OCaml parser and
our interpreter. The resulting Coq file is over 15000 lines long
and takes almost a megabyte of disk space!

\section{Correctness proof for \lambdajs{} desugaring}

In order to establish correctness, it was first needed to create
a set of predicates specifying the relationship between 
JavaScript and \lambdajs{} language features. As \lambdajs{} was
designed to implement JavaScript, the relationship is fairly
simple and natural. One place where the two languages differ
is the object heap, which is used in \lambdajs{} to implement
several JavaScript language features, including the
JavaScript lexical environment. To manage this, we use a bisimulation
relation, which relates JavaScript objects and environment records
to \lambdajs{} objects.

For the proof, we also need to keep track of certain invariants,
which state that, e.g., the JavaScript objects and their
\lambdajs{} representations are correctly formed. 
The invariants are split into two relations, where one
involves the evaluation contexts, and the other -- the heaps.
Both of these involve the bisimulation relation, but the
context invariant uses it only positively. As the bisimulation
relation only grows during the evaluation (as new objects
are created), this makes it possible to keep the proofs simple.

The correctness proof is structured using well-founded 
induction on the depth of the \lambdajs{} pretty-big-step
semantics derivation. This grants a lot of freedom in structuring
the proof, which is important because of certain structural
differences between JSCert and \lambdajs{}, which involve,
e.g., handling of ES5 references. We use three kinds of induction
hypotheses: for JavaScript statements, expressions and
function calls. The separate induction hypothesis for function
calls was required because of the complexity of calling functions
in JavaScript, and also because function calls can happen in
many possible places in JavaScript, including every
object to primitive conversion.

We prove many lemmas about the different language constructs
of JavaScript, including the internal ``specification functions''
(e.g., type conversions, operations on objects and environment
records, etc.). Typically, the proof for a lemma consists of
a series of forward reasoning steps and case analyses, and ends with
constructing a JSCert semantics derivation and proving that
the invariants were preserved. The forward reasoning is largely
automated using domain-specific tactics; the JSCert derivation
is constructed using the \texttt{eauto} tactic with a specialized
hint database (\texttt{nocore} is used to avoid using the standard
hints coming from Coq and Chargu\'eraud's TLC
library\footnote{\url{http://www.chargueraud.org/softs/tlc/}}, which is
used both by JSCert and our development).

\section{Conclusion}

In the course of our work, we found numerous cases where
\lambdajs{} did not faithfully follow the ES5 
specification\footnote{\url{https://github.com/tilk/LambdaCert/wiki/Issues-with-LambdaS5}}.
Some of these issues were serious enough that
far-reaching changes in the \lambdajs{} environment were needed
to solve them. Interestingly, no serious issues in JSCert were
found, which strengthens our confidence that it accurately models
the ES5 specification.

While proving the correctness of while loops, we found an interesting
inconsistency between the ES5 specification and the behavior
of the two most popular JavaScript implementations, Google's V5 and
Mozilla's SpiderMonkey. The discovery of the inconsistency,
which involves exceptions and statement result values,
resulted in a correction being made in the official ES6
specification\footnote{\url{https://esdiscuss.org/topic/loop-unrolling-and-completion-values-in-es6}}.

The proof scripts we developed are very taxing for the Coq
proof assistant -- the compilation of the whole development
requires several gigabytes of RAM memory, and takes several
hours to compile, even though efforts were made to optimize it.
We are still looking for a solution for the performance issues.

The development is still not complete, but it already
includes most of the important core features of the
JavaScript language. We are looking forward to receiving
feedback from the community.

%\appendix
%\section{Appendix Title}

%This is the text of the appendix, if you need one.

%\acks

%Acknowledgments, if needed.

% We recommend abbrvnat bibliography style.

\bibliographystyle{abbrvnat}

\bibliography{mybib}

% The bibliography should be embedded for final submission.

%\begin{thebibliography}{}
%\softraggedright

%\bibitem[Smith et~al.(2009)Smith, Jones]{smith02}
%P. Q. Smith, and X. Y. Jones. ...reference text...

%\end{thebibliography}


\end{document}

%                       Revision History
%                       -------- -------
%  Date         Person  Ver.    Change
%  ----         ------  ----    ------

%  2013.06.29   TU      0.1--4  comments on permission/copyright notices

